\documentclass{article}
\usepackage{graphicx}
\usepackage{hyperref} % Recommended for references

% Set page size and margins to approximate the 65 lines per page
% This is an approximation, exact line count depends on font, spacing, etc.
\usepackage[a4paper, total={6in, 8in}, top=1in, bottom=1in, left=1in, right=1in]{geometry}
\linespread{1.1} % Adjust line spacing if needed to fit ~65 lines per page

\begin{document}

\section{Introduction}
This is the introduction section of the paper. It provides context and background for the work presented. We discuss the problem domain and the motivation behind our approach. The goal is to clearly state what the paper aims to achieve and its significance. We will outline the structure of the remaining sections.

\section{Methodology}
In this section, we detail the proposed methodology. We describe the different components of our system and how they interact. The system pipeline is intended to be illustrated in Figure \ref{fig:pipeline}, showing the flow of data and processing steps. We explain the rationale behind the design choices made and how they contribute to solving the problem.

\begin{figure}[htbp]
    \centering
    % This is a placeholder image. A detailed figure illustrating the pipeline
    % or example outputs will replace this placeholder in the final version.
    \includegraphics[width=0.8\textwidth]{placeholder.png} % Assuming placeholder.png exists
    \caption{Placeholder Figure. This figure is intended to illustrate the overall pipeline workflow, showing the sequence of steps from input data to final output, or example outputs generated by the system.}
    \label{fig:pipeline}
\end{figure}

We further elaborate on the specific algorithms and techniques used in each stage of the pipeline. Details regarding data preprocessing, model architecture, training procedures, and evaluation metrics are provided.

\section{Experiments and Results}
This section presents the experimental setup and the results obtained. We describe the dataset used for evaluation and the metrics employed to measure performance. We compare our approach with existing methods and analyze the results.

\section{Discussion}
We discuss the implications of our findings and the strengths and limitations of our approach. Potential areas for future work are also identified.

\section{Conclusion}
Finally, we summarize the main contributions of this paper and reiterate the significance of our work.

% The main text ends here. The references will follow.
% The current main text content (including sections, text, and figure space)
% is significantly less than 4 pages (approx 260 lines).
% The figure environment adds vertical space equivalent to several lines of text.
% The total lines used for main text content + figure space is well within the limit.

\newpage % Start references on a new page

\begin{thebibliography}{99}

% Example references (replace with actual references)
\bibitem{ref1} Author, A. (Year). Title of the paper. \textit{Journal Name}, Volume(Issue), Pages.
\bibitem{ref2} Author, B. (Year). \textit{Title of the Book}. Publisher.

\end{thebibliography}

\end{document}